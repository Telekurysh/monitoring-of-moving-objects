\maketableofcontents

\chapter*{ВВЕДЕНИЕ}
\addcontentsline{toc}{chapter}{Введение}

В современных условиях глобализации и цифровизации ключевую роль в оптимизации бизнес-процессов играют технологии IoT (Интернета вещей) и системы мониторинга в реальном времени. Согласно исследованию MarketsandMarkets, мировой рынок IoT-решений к 2026 году достигнет 650 млрд, а объем данных, генерируемых подключенными устройствами, ежегодно растет на 30-50 млрд.

Проект SensorTrack Pro направлен на решение этих проблем. Система обеспечивает мониторинг движущихся объектов (транспорт, спецтехника, грузы) с помощью IoT-датчиков, анализирует телеметрию, отслеживает пересечение геозон и автоматически генерирует оповещения при критических событиях. Актуальность проекта подкреплена ограничениями существующих аналогов: например, 70% решений на рынке поддерживают только статические зоны, а кастомизация метрик доступна лишь в 15% случаев.

Целью курсовой работы является разработка информационной системы для мониторинга объектов в реальном времени с интеграцией IoT-устройств и аналитическим функционалом. Для достижения цели поставлены следующие задачи:

\begin{itemize}
	\item провести анализ существующих систем мониторинга и выявить их недостатки;
	\item спроектировать архитектуру базы данных с поддержкой геопространственных данных;
	\item рассмотреть модели баз данных и выбрать подходящую;
	\item проанализировать существующие СУБД и выбрать нужную;
 	\item спроектировать и разработать БД;
    \item спроектировать и разработать WEB-интерфейс.
\end{itemize}
