\begin{essay}{РЕФЕРАТ}

Данная курсовая работа представляет собой разработку информационной системы SensorTrack Pro, предназначенной для мониторинга движущихся объектов (транспорт, грузы, техника) в реальном времени с использованием IoT-датчиков. Система анализирует телеметрические данные, отслеживает пересечение геозон и автоматически генерирует оповещения при критических событиях, таких как выход объекта за установленные границы или отклонение от маршрута. Решение ориентировано на оптимизацию логистических процессов, повышение безопасности и снижение рисков потери грузов.

Для реализации проекта выбран следующий технологический стек:

Backend: Python 3 с использованием асинхронного фреймворка FastAPI для создания API.
База данных: PostgreSQL
Frontend: HTML и CSS для построения интерфейса.

Ключеые слова: мониторинг, геозоны, FastAPI, PostgreSQL, телеметрия.

\end{essey}