\newpage
\chapter{Аналитическая часть}

\section{Cуществующие решения}

Учитывая стремительный рост рынка IoT-решений и высокий спрос на автоматизацию логистических процессов, на рынке уже существуют решения, предоставляющие различный функционал.

Рассмотрим только самые популярные из них, такие как:

\begin{itemize}
	\item FleetMind;
	\item GeoGuard;
	\item TrackFlow;
\end{itemize}

Выделим следующие критерии для сравнения выбранных решений:

\begin{enumerate}
	\item гибкость зон;
	\item типы метрик;
	\item ролевая модель;
\end{enumerate}


Результаты сравнения выбранных решений по заданным критериям представлены в таблице \begin{table}[H]
	\centering
	\caption{Сравнение существующих решений}
	\label{tbl:compare_realizations}
	\begin{tabular}{|l|l|l|l|}
		\hline
		\textbf{\makecell{Критерий}} & \textbf{\makecell{FleetMind}} & \textbf{\makecell{GeoGuard}} & \textbf{\makecell{TrackFlow}} \\ \hline
		
		\makecell{Типы \\ поддерживаемых \\ зон} & \makecell{Только \\ круги} & \makecell{Только \\ прямоугольники} & \makecell{Статичные \\ зоны} \\ \hline
		
		\makecell{Типы \\ метрик} & \makecell{Фиксированный \\ набор метрик} & \makecell{Ограниченные \\ метрики} & \makecell{Базовые \\ метрики} \\ \hline
		
		\makecell{Ролевая \\ модель} & \makecell{2 роли \\ (админ/оператор)} & \makecell{1 роль \\ (админ)} & \makecell{2 роли \\ без настройки прав} \\ \hline
		
	\end{tabular}
\end{table}

Таким образом, ни одно из трех рассмотренных решений не удовлетворяет всем критериям сравнения. Также стоит отметить, что все они являются зарубежными, отечественные аналоги либо отсутствуют, либо слишком непопулярны.

\section{Формализация задачи}

В рамках курсовой работы требуется разработать информационную систему для мониторинга движущихся объектов (транспорт, грузы, спецтехника) с использованием IoT-датчиков. Система должна включать:

\begin{enumerate}
\item Базу данных для хранения
\begin{itemize}
\item Данных о движущихся объектах (ID, тип, статус, привязанные датчики).
\item Параметров геозон (координаты, тип зоны: круг/прямоугольник, радиус/границы).
\item Телеметрии (время, местоположение, скорость, кастомизируемые метрики).
\item Пользователей с ролевой моделью (администратор, оператор, аналитик).
\item Событий (пересечение зон, критические показатели датчиков).
\end{itemize}
\item Веб-приложение с функционалом
\begin{itemize}
\item Реального времени: отображение позиций объектов на карте, оповещения о событиях.
\item Управления геозонами: создание, редактирование, удаление зон.
\item Аналитики: генерация отчетов по историческим данным (графики потребления топлива, статистика нарушений маршрутов).
\end{itemize}
\item Требования к системе
\begin{itemize}
\item Поддержка кастомизируемых метрик (например, температура, уровень топлива) через ключ-значение.
\end{itemize}
\item Разграничение прав доступа
\begin{itemize}
\item Администратор: управление пользователями, объектами, зонами.
\item Оператор: мониторинг данных и реагирование на события.
\item Аналитик: работа с отчетами и историческими данными.
\end{itemize}
\item Дополнительные задачи
\begin{itemize}
\item Реализация API для интеграции с внешними IoT-устройствами.
\item Оптимизация запросов к БД для обработки данных в реальном времени.
\item Тестирование системы на устойчивость к пиковым нагрузкам.
\end{itemize}
\end{enumerate}

\clearpage

\section{Формализация данных}

Разрабатываемая база данных предназначена для хранения информации о движущихся объектах, датчиках, событиях, зонах, пользователях и их взаимодействии.

\begin{table}[H]
	\centering
	\caption{Категории данных в БД и их атрибуты}
	\label{tbl:sensortrack_data}
	\begin{tabular}{|l|l|}
		\hline
		\textbf{\makecell{Категория}} & \textbf{\makecell{Атрибуты}} \\ \hline
		
		\makecell{Объект \\ (Objects)}         & \makecell{ID объекта, название, тип объекта, \\ дата создания} \\ \hline
		\makecell{Датчик \\ (Sensors)}         & \makecell{ID датчика, ID объекта, тип датчика, \\ местоположение, статус, дата установки} \\ \hline
		\makecell{Событие \\ (Events)}         & \makecell{ID события, ID датчика, временная метка, \\ широта, долгота, скорость, тип события} \\ \hline
		\makecell{Зона \\ (Zones)}             & \makecell{ID зоны, название зоны, координаты границ, \\ тип зоны (круг/прямоугольник)} \\ \hline
		\makecell{Оповещение \\ (Alerts)}      & \makecell{ID оповещения, ID события, тип оповещения, \\ уровень критичности, сообщение, дата создания} \\ \hline
		\makecell{Маршрут \\ (Routes)}         & \makecell{ID маршрута, ID объекта, время начала, \\ время окончания, статус маршрута} \\ \hline
		\makecell{Телеметрия \\ (Telemetry)}   & \makecell{ID записи, ID объекта, временная метка, \\ уровень заряда батареи, температура, \\ уровень сигнала} \\ \hline
		\makecell{Пользователь \\ (Users)}     & \makecell{ID пользователя, логин, хеш пароля, \\ роль (администратор/оператор/аналитик)} \\ \hline
		\makecell{Связь \\ объект-зона \\ (ObjectZones)} & \makecell{ID объекта, ID зоны, \\ время входа в зону, время выхода из зоны} \\ \hline
		\makecell{Связь \\ пользователь-объект \\ (UserObjects)} & \makecell{ID пользователя, ID объекта, \\ уровень доступа} \\ \hline
	\end{tabular}
\end{table}
\clearpage
Также на рисунке \ref{fig:er} изображена ER-диаграмма системы в нотации Чена.

\imgs{er}{h!}{0.18}{ER-диаграмма в нотации Чена}

\clearpage

\section{Формализация категорий пользователя}

Для взаимодействия с системой SensorTracker Pro выделено три категории пользователей: оператор, аналитик и администратор. Каждая роль предоставляет уникальный набор прав, соответствующий задачам управления мониторингом объектов.

Оператор имеет доступ к базовому функционалу мониторинга:
\begin{itemize}
	\item Просмотр позиций объектов на карте в режиме реального времени.
	\item Отслеживание событий (пересечение зон, критические показатели датчиков).
	\item Управление оповещениями: подтверждение, закрытие, отправка уведомлений.
	\item Фильтрация объектов по типу, статусу или привязанным зонам.
\end{itemize}

Аналитик работает с историческими данными и аналитикой:
\begin{itemize}
	\item Генерация отчетов по метрикам (расход топлива, средняя скорость, нарушения маршрутов).
	\item Экспорт отчетов в форматах CSV.
\end{itemize}

Администратор обладает полным контролем над системой:
\begin{itemize}
	\item Управление пользователями: создание, удаление, назначение ролей.
	\item Настройка геозон (добавление, изменение границ, привязка к объектам).
	\item Конфигурация метрик: создание кастомизируемых параметров (например, температура, давление).
	\item Доступ к журналам событий и аудиту действий пользователей.
\end{itemize}

\imgs{use_case_diagram}{h!}{0.05}{Use-case Диаграмма}


\clearpage

\section{Модели баз данных}

База данных — это структурированная совокупность данных, организованная для эффективного хранения, обработки и доступа к информации. В контексте IoT-систем, таких как SensorTracker Pro, базы данных играют ключевую роль в управлении потоками телеметрии, событиями и геопространственными данными.

Модель базы данных определяет принципы организации данных, их взаимосвязи и методы взаимодействия. Выбор модели напрямую влияет на производительность системы, масштабируемость и поддержку сложных запросов.

\subsection{Основные типы моделей}
\begin{itemize}
	\item Дореляционные модели — иерархическая и сетевая;
	\item Реляционные модели — табличные структуры с отношениями;
	\item Постреляционные модели — NoSQL, графовые, документно-ориентированные.
\end{itemize}

\subsection{Дореляционные модели}
Дореляционные подходы, такие как иерархическая и сетевая модели, устарели для современных IoT-решений.

Иерархическая модель организует данные в древовидные структуры («родитель-потомок»), но не поддерживает связи «многие-ко-многим», что критично для отслеживания объектов в нескольких зонах.
Сетевая модель позволяет множественные связи, но её сложно адаптировать под динамически изменяемые метрики (например, добавление новых типов датчиков).
Обе модели обладают низкой структурной гибкостью, что делает их непригодными для систем, требующих частых изменений схемы данных.

\subsection{Реляционные модели}
Реляционные базы данных (РБД) организуют данные в таблицы с четкими связями. Для SensorTracker Pro это обеспечивает:

Целостность данных: транзакции ACID гарантируют корректность операций (например, обновление статуса объекта при пересечении зоны).
Гибкость запросов: SQL позволяет агрегировать данные (средняя скорость объекта, частота событий) и работать с геопространственными расширениями (PostGIS).
Структурированность: таблицы Objects, Sensors, Events связаны через внешние ключи, что упрощает анализ маршрутов и оповещений.
Пример использования: хранение координат объектов в формате WKT (Well-Known Text) для визуализации на карте.

\subsection{Постреляционные модели}
NoSQL-решения (документные, ключ-значение) подходят для обработки больших потоков данных в реальном времени, но имеют ограничения:

Документные БД (MongoDB) удобны для хранения JSON-данных телеметрии, но сложны для выполнения JOIN-запросов между объектами и зонами.
Графовые БД (Neo4j) эффективны для анализа связей, но избыточны для задач мониторинга.
Временные ряды (InfluxDB) оптимизированы под метрики, но не поддерживают геоданные.
\section{Вывод}
Для SensorTracker Pro выбрана реляционная модель (PostgreSQL + PostGIS), так как она:

Обеспечивает целостность данных при высокой нагрузке (до 1000+ объектов).
Поддерживает геопространственные запросы (поиск объектов в зоне, расчет расстояний).
Позволяет гибко настраивать связи между сущностями (например, ObjectZones для отслеживания пересечений).
Интегрируется с аналитическими инструментами (генерация отчетов, визуализация на карте).
Альтернативные модели не покрывают все требования системы: NoSQL не гарантирует ACID, а дореляционные подходы устарели для IoT-решений.